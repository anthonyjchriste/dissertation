\begin{abstract}
	Distributed Sensor Networks (DSNs) are faced with a myriad of technical challenges. This dissertation examines two important DSN challenges. 
	
	One problem that is apparent in any DSN is converting ``primitive" sensor data into actionable products and insights. For example, a DSN for power quality (PQ) might gather primitive data in the form of raw voltage waveforms and produce actionable insights in the form of classified power quality events such as voltage sags or frequency swells or provide the ability to predict when PQ events are going to occur by observing cyclical data. For another example, a DSN for infrasound might gather primitive data in the form of microphone counts and produce actionable insight in the form of determining what, when, and where the signal came from. To make progress towards this problem, DSNs typically implement one or more of the following strategies: detecting signals in the primitive data (deciding if something is there), classification of signals from primitive data (deciding what is there), localization of signals (when and where did the signals come from), and by forming relationships between primitive data by finding correlations between spatial attributes, temporal attributes, and by associating metadata with primitive data to provide contextual information not collected by the DSN. These strategies can be employed recursively. As an example, the result of aggregating typed primitive data provides a new higher level of types data which contains more context than the data from which is was derived from. This new typed data can itself be aggregated into new, higher level types and also participate in relationships.
	
	A second important challenge is managing data volume. Most DSNs produce large amounts of (increasingly multimodal) primitive data, of which only a tiny fraction (the signals) is actually interesting and useful. The DSN can either utilize one of two strategies: keep all of the information and primitive data forever, or employ some sort of strategy for systematically discarding (hopefully uninteresting and not useful) data. As sensor networks scale in size, the first strategy becomes unfeasible. Therefore, DSNs must find and implement a strategy for managing large amounts of sensor data. The difficult part is finding an effective and efficient strategy deciding what data is interesting and must be kept and what data to discard.
	
	This dissertation investigates the design, implementation, and evaluation of the Laha framework, which is intended to address both of these problems. First, the Laha framework provides a multi-leveled representation for structuring and processing DSN data. The structure and processing at each level is designed with the explicit goal of turning low-level data into actionable insights. Second, each level in the framework implements a ``time-to-live" (TTL) strategy for data within the level. This strategy states that data must either ``progress" upwards through the levels towards more abstract, useful representations within a fixed time window, or be discarded and lost forever. The TTL strategy is interesting because when implemented, it allows DSN designers to calculate upper bounds on data storage at each level of the framework and supports graceful degradation of DSN performance.
	
	There are several smaller, but still important problems that exist within the context of these two larger problems. These larger problems are addressed by solving a series of smaller problems. Examples of the smaller problems that Laha hopes to overcome in transit to the larger goals include optimization of triggering, detection, and classification, building a model of sensing field topology, optimizing sensor energy use, optimizing bandwidth, and providing predictive analytics for DSNs.
	
	The claim of this dissertation is that the Laha Framework provides a generally useful representation for DSNs. I evaluate this claim in the following ways.
	
	First, to evaluate the generality of the network, I designed, implemented, and deployed two Laha-compliant reference networks in two different domains, power quality and infrasound. These  reference implementations generate evidence for the ways in which Laha supports the goals of the sensor networks and ways in which it falls short. The implementations also provide insights into the types of distributed sensor networks for which Laha is well-suited, and the types for which it is not.
	
	Second, these implementations enable me to evaluate the multi-level representation system. I claim that Laha enables a distributed sensor network to derive actionable insights from low level data, and that each of the levels is important to that process. The two reference implementations provide concrete data as to the set of levels that are useful in practice, or whether different levels would be more appropriate, or if the level strategy itself has problematic features.
	
	Third, my evaluation assessed the TTL-based approach to managing data volume. I claim that a benefit of Laha's mechanism for managing data is that it enables the calculation of upper bounds on data storage requirements. In my evaluation, I developed the analytical procedures required for calculating data storage requirements, and determined if these procedures are valid in practice.  One obvious problem with a TTL approach is the possibility of false negatives: data that is discarded before it has been recognized as important. My evaluation includes studies designed to assess the frequency of false negatives and how important the problem is in practice.
	
	Finally, my evaluation assesses the ability to solve the tertiary problems of optimizing triggering, detection, classification, bandwidth, sensor energy usage, predictive analytics, and the ability to build a model of the sensing field. I claim that these problems need to be addressed in some form in order to solve the larger problems of turning primitive data into actionable insights and to provide a mechanism for managing large amounts of sensor data. I compare and contrast state of the art algorithms present in the literature to determine if they are effective in practice and useful for addressing the two larger problems. My evaluation provide metrics on false negatives and false positives as a means of demonstrating effectiveness.
\end{abstract}